\documentclass[12pt]{article}
\usepackage[utf8]{inputenc}
\usepackage[round]{natbib}
\setlength{\parindent}{2em}
\setlength{\parskip}{0.1em}
\renewcommand{\baselinestretch}{1.3}
\usepackage[margin=1in]{geometry}
\usepackage{amssymb,amsmath,amsfonts,eurosym,ulem,graphicx,color,setspace,sectsty,comment,footmisc,natbib,pdflscape,array,hyperref,pdflscape,array,makecell, threeparttable, booktabs, geometry, xcolor, caption, subcaption, placeins,adjustbox}
\usepackage[compact]{titlesec}
\usepackage{changepage}
\usepackage{enumitem}

\usepackage{amssymb, amsthm, amsmath}
\usepackage{xparse}
\usepackage{verbatim}

\newtheorem{thm}{Theorem}
\newtheorem*{thm*}{Theorem}
\providecommand*\thmautorefname{Theorem}

% Lemmata
\newtheorem{lemma}[thm]{Lemma}
\newtheorem*{lemma*}{Lemma}
\providecommand*\lemmaautorefname{Lemma}

\newtheorem{theorem}{Theorem}
\newtheorem{proposition}[theorem]{Proposition}
\newtheorem{corollary}[theorem]{Corollary}
\newtheorem{assumptionp}[theorem]{Assumption}


\newenvironment{assumptionalt}[1]{
  \renewcommand\theassumptionalt{#1}
  \assumptionalt
}{\endassumptionalt}

\usepackage{hyperref}
\hypersetup{
    colorlinks=true,
    linkcolor=purple,
    filecolor=purple,
    urlcolor=purple,
    citecolor=purple}
%\newcommand{\todo}[1]{\emph{\color{red}{#1}}}
%\ \todo and \note on, uncomment to turn off
 \newcommand{\todo}[1]{}
\urlstyle{same}
\title{\textbf{Work Documentation: \\ Climate Data in Egypt}}


\author{
  Till Meissner}
\date{\today}

\newcolumntype{P}[1]{>{\centering\arraybackslash}p{#1}}

\begin{document}
\maketitle
\section{Overview}

The work performed under this assignment can be divided into three steps:
\begin{enumerate}
    \item \textbf{Literature review and screening existing documents to identify relevant metrics and datasets to capture climate stress} at the governorate and household level. The household surveys have identified heat stress and drought as relevant phenomena of slow onset climate change in Upper Egypt.
    \item \textbf{Downloading, preprocessing, and aggregating climate data} at the governorate and household levels using Google Earth Engine, Copernicus Climate Data Store and RStudio. Beyond climate data, I gathered data on household altitude and slopes derived from Digital Elevation Models, distance to the Nile and other waterways based on data from Open Street Map as well as distances to nearest cities through the Google Distance Matrix API.
    \item \textbf{Providing evidence for mounting climate stress} in Upper Egypt in the run up and since the start of the intervention. I did this by creating a number of graphs and figures (e.g.
    \item \textbf{Creating panel datasets} Finally, I created panel datasets at the governorate and household level. I also  created a set of vulnerability indices through inverse distance weighting, capturing exposure to climate stress at the household level.
\end{enumerate}

\section{Capturing Climate Stress: Metrics and Data Considerations}

A review of the literature suggests that (daily/monthly/yearly deviations from historical climatology, changes in average mean, maximum and minimum temperatures; frequency, duration, and intensity of heat waves; number of winter days with maximum temperatures above 30° Celsius.)
I identified ERA5 reanalysis data produced by the European Centre for Medium-Range Weather Forecasts (ECMWF) as part of the Copernicus Climate Change Service (C3S) as the most appropriate data source (see Metrics and Data Considerations Section for more information) to capture changes in temperature and heat stress. For drought occurrence, the Evaporative Stress Index (ESI) from the National Oceanic and Atmospheric Administration (NOAA) as well as the Normalised Difference Vegetation Index (NDVI) and the Enhanced Vegetation Index (EVI) were identified as relevant metrics.


\begin{enumerate}
    \item Why ERA5 data? Heat stress is a multifaceted phenomenon, to capture this, I constructed a number of metrics, many of which require daily temperature data. The literature distinguishes between different types of metrics to capture heat stress. I capture the following three:
    \begin{itemize}
        \item Measured Air Temperature
        \item Heat Index: A combination of Temperature + Relative Humidity.
        \item Universal Thermal Climate Index (UTCI): The Universal Thermal Climate Index, UTCI, is a bioclimatic index for describing the physiological comfort of the human body under specific meteorological conditions (Bröde et al. 2012). It takes into account not just the ambient temperature but also other variables like humidity, wind and radiation, all factors significantly affecting our physiological reaction to the surrounding environment.
    \end{itemize}
    \item As the figure shows, difference in perceived heat are amplified especially in summer.

\begin{figure}
    \centering
    \includegraphics[scale = 0.6]{Graphs/temperature_comparingmeasures.pdf}
\end{figure}

    \item Regarding \textbf{drought}, the following types of droughts can be distinguished:
    \begin{itemize}
        \item Meteorological drought/Lack of Precipation: This is a commonly used measure in the literature, often captured by the Standardised Precipitation Index (SPI) or the Standardised Evapotranspiration Index (SPEI). For Upper Egypt, due to its historically low rainfall, they are however, not defined and could therefore not be used for the study.
        \item Agricultural drought (crop stress): Instead, I capture drought by using the Evaporative Stress Index (ESI)
        \item Hydrological drought: (low water flows in rivers) --> need to look at! Could be the Data from SWOT?
    \end{itemize}
\end{enumerate}

\section{Documentation}
\begin{enumerate}
    \item \textbf{Construct grid-level dataset and associated panel of climate variables}
    \begin{itemize}
        \item These should \textbf{include all variables mentioned above} (temperature, precipitation, humidity, air pressure etc.), constructed variables (number of hot days, season onset, heatwaves) and literature-based measures (e.g. climate stress).
        \item Include these variables from 2010-date, with year-season frequency.
        \item Additionally, include topology (elevation, slope) \item Construct a measure of proximity to the Nile for each grid cell.
        \item I would also be interested in some measure of ``road" or ``market" access to measure how easy it is (or how long it takes) to reach certain locations.
    \end{itemize}
        \item Set-up infrastructure to \textbf{merge this with our household survey.}
        \begin{itemize}
            \item I will share an example of our HH GPS data.
        \end{itemize}
        \item \textbf{Research crop- and livestock-specific measures for climate stress/vulnerability.} Ideally, we want to know how to map climate data into a measure of vulnerability for a household, conditional on which crops/livestock they rely on as an income source.
        \begin{itemize}
            \item The previous RA folder provides some information on how this could be done (e.g. through thresholds of temperature at which yields fall)
            \item As a first step, it would be good to visualize thresholds of vulnerability (e.g. in terms of temperature or an index of heat) for different livestock and crops commonly found in Egypt -- this will demonstrate that there is variation in climate vulnerability stemming from HH occupation and the program's livestock transfer.
        \end{itemize}
\end{enumerate}

\end{document}


